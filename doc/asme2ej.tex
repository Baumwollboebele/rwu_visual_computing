%%%%%%%%%%%%%%%%%%%%%%%%%%% asme2ej.tex %%%%%%%%%%%%%%%%%%%%%%%%%%%%%%%
% Template for producing ASME-format journal articles using LaTeX    %
% Written by   Harry H. Cheng, Professor and Director                %
%              Integration Engineering Laboratory                    %
%              Department of Mechanical and Aeronautical Engineering %
%              University of California                              %
%              Davis, CA 95616                                       %
%              Tel: (530) 752-5020 (office)                          %
%                   (530) 752-1028 (lab)                             %
%              Fax: (530) 752-4158                                   %
%              Email: hhcheng@ucdavis.edu                            %
%              WWW:   http://iel.ucdavis.edu/people/cheng.html       %
%              May 7, 1994                                           %
% Modified: February 16, 2001 by Harry H. Cheng                      %
% Modified: January  01, 2003 by Geoffrey R. Shiflett                %
% Use at your own risk, send complaints to /dev/null                 %
%%%%%%%%%%%%%%%%%%%%%%%%%%%%%%%%%%%%%%%%%%%%%%%%%%%%%%%%%%%%%%%%%%%%%%

% Hoch 3 beispielbilder
% hoch 1, schräg 2 weitere



%%% use twocolumn and 10pt options with the asme2ej format
\documentclass[twocolumn,10pt]{asme2ej}

\usepackage{graphicx} %% for loading jpg figures
\usepackage{hyperref}   % to set up hyperlinks
\hypersetup{
	colorlinks=true,
	linkcolor=blue,
	citecolor=blue,
	urlcolor=blue,
}
\usepackage[square,numbers]{natbib}
\usepackage{float}
\usepackage{tikz}
\newcommand*\redcircled[1]{\tikz[baseline=(char.base)]{
            \node[shape=circle,draw,inner sep=2pt, fill=red!40] (char) {#1};}}
\newcommand*\bluecircled[1]{\tikz[baseline=(char.base)]{
            \node[shape=circle,draw,inner sep=2pt, fill=blue!40] (char) {#1};}}


% \usepackage{url}
% \usepackage{breakurl}
% \usepackage[breaklinks]{hyperref}   
%\def\UrlBreaks{\do\/\do-}

%% The class has several options
%  onecolumn/twocolumn - format for one or two columns per page
%  10pt/11pt/12pt - use 10, 11, or 12 point font
%  oneside/twoside - format for oneside/twosided printing
%  final/draft - format for final/draft copy
%  cleanfoot - take out copyright info in footer leave page number
%  cleanhead - take out the conference banner on the title page
%  titlepage/notitlepage - put in titlepage or leave out titlepage
%  
%% The default is oneside, onecolumn, 10pt, final


\title{Computer Vision SS22 Assignment 1: Document Scanner}

%%% first author
\author{Felix Hamburger
    \affiliation{
	Student ID: 35925\\
	Computer Vision SS22\\
	Computer Science Master\\
	Ravensburg Weingarten University\\
    Email: felix.hamburger@rwu.de
    }	
}

%%% second author
%%% remove the following entry for single author papers
%%% add more entries for additional authors
\author{Mario Amann
    \affiliation{ 
    Student ID: 35926\\
    Computer Vision SS22\\
    Computer Science Master\\
    Ravensburg Weingarten University\\
    Email: mario.amann@rwu.de
     }	
}



\begin{document}

\maketitle   

% Notes:
% Citations
% https://answers.opencv.org/question/32411/opencv-bibtex-citation/
% http://citebay.com/how-to-cite/opencv/

% Functions:

% imread: https://docs.opencv.org/3.4/d4/da8/group__imgcodecs.html#ga288b8b3da0892bd651fce07b3bbd3a56

% cvtColor: https://docs.opencv.org/3.4/de/d25/imgproc_color_conversions.html
% RGB[A] to Gray:Y←0.299⋅R+0.587⋅G+0.114⋅B


%%%%%%%%%%%%%%%%%%%%%%%%%%%%%%%%%%%%%%%%%%%%%%%%%%%%%%%%%%%%%%%%%%%%%%
\begin{abstract}
{\it This paper presents a usecase for a DIN A4 document scanner. 
The explanations contained in this paper will be divided into three steps.
First the corner points of the document will be detected utilizing the canny edge corner detection.
By performing a perspective transofmation the document will then be provided in a top-view.
With the help of a filter the document will then be changed into a binary format.
}
\end{abstract}

%%%%%%%%%%%%%%%%%%%%%%%%%%%%%%%%%%%%%%%%%%%%%%%%%%%%%%%%%%%%%%%%%%%%%%
%\begin{nomenclature}
% \entry{A}{You may include nomenclature here.}
%\entry{$\alpha$}{There are two arguments for each entry of the nomemclature environment, the symbol and the definition.}
% \end{nomenclature}

%The primary text heading is  boldface and flushed left with the left margin.  The spacing between the  text and the heading is two line spaces.

%%%%%%%%%%%%%%%%%%%%%%%%%%%%%%%%%%%%%%%%%%%%%%%%%%%%%%%%%%%%%%%%%%%%%%
\section{Introduction}
\noindent
In everyday life, there are often situations in which it is advantageous to scan a document that has
been received and keep it as a copy. Unfortunately not everyone has a ready-to-use scanner with them to scan the document.
Though most of the people have a device capable of doing just this - a cell phone with a camera.
Using various methods, it is possible to scan documents with the help of the cell phone camera, similar to a scanner, 
and save them in a suitable foramt.
This paper will give your step by step guide on how it is possible to implement these methodologies. 

% This article illustrates preparation of ASME paper using \LaTeX2\raisebox{-.3ex}{$\epsilon$}. The \LaTeX\  macro \verb+asme2ej.cls+, the {\sc Bib}\TeX\ style file \verb+asmems4.bst+, and the template \verb+asme2ej.tex+ that create this article are available on the WWW  at the URL address \url{http://iel.ucdavis.edu/code/}. To ensure compliance with the 2003 ASME MS4 style guidelines  \cite{asmemanual}, you should modify neither the \LaTeX\ macro \verb+asme2ej.cls+ nor the {\sc Bib}\TeX\ style file \verb+asmems4.bst+. By comparing the output generated by typesetting this file and the \LaTeX2\raisebox{-.3ex}{$\epsilon$} source file, you should find everything you need to help you through the preparation of ASME paper using \LaTeX2\raisebox{-.3ex}{$\epsilon$}. Details on using \LaTeX\ can be found in \cite{latex}. 

% In order to get started in generating a two-column version of your paper, please format the document with 0.75in top margin, 1.5in bottom margin and 0.825in left and right margins.  Break the text into two sections one for the title heading, and another for the body of the paper.  

% The format of the heading is not critical, on the other hand formatting of the body of the text is the primary goal of this exercise.  This will allow you to see that the figures are matched to the column width and font size of the paper.  The double column of the heading section is set to 1.85in for the first column, a 0.5in spacing, and 4.5in for the second column.  For the body of the paper, set it to 3.34in for both columns with 0.17in spacing, both are right justified. 

% The information that is the focus of this exercise is found in 
% section~\ref{sect_figure}.
% Please use this template to format your paper in a way that is similar to the printed form of the Journal of Mechanical Design.  This will allow you to verify that the size and resolution of your figures match the page layout of the journal.  The ASME Journal of Mechanical Design will no longer publish papers that have the errors demonstrated here.

% ASME simply requires that the font should be the appropriate size and not be blurred or pixilated, and that lines should be the appropriate weight and have minimal, preferably no, pixilation or rasterization.

% The journal uses 10pt Times Roman Bold for headings, but Times Bold is good enough for this effort.  The text is set at 9pt Times Roman, and again Times will be fine.  Insert a new line after the heading, and two lines after each section.  This is not exactly right but it is close enough.


%%%%%%%%%%%%%%%%%%%%%%%%%%%%%%%%%%%%%%%%%%%%%%%%%%%%%%%%%%%%%%%%%%%%%%
\section{Preprocessing}
\label{section:preprocessing}
% grayscale_image = cvtColor(image,COLOR_BGR2GRAY)
% blurred_image = GaussianBlur(grayscale_image, ksize=(5,5),sigmaX=0,sigmaY=0)
% Blurred picture

The first step is to read the image (Figure \ref{fig:orginial}). This is done with the imread function of openCV which 
outputs the image as numpy array with the shape (Height, Width, Channel).
By default the decoded image is stored in the BGR-format. Which means, the first channel contains the
blue color channel, the second one contains the green color channel and the last the red one.
\begin{figure}[H]
    \centerline{\includegraphics[width=2.5in]{output/hoch_3_1_original.jpg}}
    \caption{Original image.}
    \label{fig:orginial}
\end{figure}
The original image is seperately stored, because for further steps the photo is needed in a grayscale format (Figure \ref{fig:grayscale}).
This convertion is made with the cvtColor\cite{opencv_cvtColor} function of openCV. The convertion of colorspace is made with the following
formula\cite{opencv_rgb2gray}:
\begin{center}
    $Y = B * 0.114 + G * 0.587 + R * 0.299$
    \label{eq_rgb2gray}
\end{center}
\begin{figure}[H]
    \centerline{\includegraphics[width=2.5in]{output/hoch_3_2_grayscaleimg.jpg}}
    \caption{Grayscale image.}
    \label{fig:grayscale}
\end{figure}
The next preprocessing step includes the smoothing/blurring of the grayscale image.
This is done to remove noise from the image\cite{opencv_smoothingimages}. In our case
the Gaussian Blurring\cite{opencv_gaussianblur} is used. The image is convolved with the Gaussian Kernel.
As size of the kernel the size (5,5) is used.
Because a document could be landscape or portrait format, 
it makes sense to use same values in x and y direction in the kernel size, as well as for the parameters
sigmaX and sigmaY.
The optimum values of these parameters were determined from the results of several scans of sample documents.
The parameters sigmaX and sigmaY are used to calculate the gaussian filter coefficients. In the function call
sigmaX and sigmaY are set to zero, which means the Gaussian Kernel computes these variables from the given
ksize\cite{opencv_getgaussiankernel}.
\begin{center}
    $\sigma = 0.3 * (0.5 * (ksize - 1 ) - 1) + 0.8$
    \label{eq_sigma}
\end{center}
The gaussian filter coefficients are computated with the following formula\cite{opencv_getgaussiankernel}:
\begin{center}
    $G_i = \alpha * e^{- \frac{(i - \frac{(ksize - 1)}{2})^2}{2*\sigma^2 } }$
    \label{eq_sigma}
\end{center}
\begin{figure}[H]
    \centerline{\includegraphics[width=2.5in]{output/hoch_3_3_gaussianblur.jpg}}
    \caption{Grayscale image with Gaussian Blur.}
    \label{fig:grayscale}
\end{figure}
% The heading is boldface with upper and lower case letters. 
% If the heading should run into more than one line, the run-over is not left-flushed.

%%%%%%%%%%%%%%%%%%%%%%%%%%%%%%%%%%%%%%%%%%%%%%%%%%%%%%%%%%%%%%%%%%%%%%
\section{Edge Detection}
\label{section:edgedetection}
% edges = Canny(blurred_image,75,180, L2gradient=True)
\noindent
After we took our preprocessing step and chose a suitable noise reduction algorithm 
we start with the edge detection.
For edge detection we use the Canny edge\cite{canny_paper} detector which is a multi-staged algorithm:
\begin{enumerate}
    \item Noise reduction.
    \item Calculating the intensity gradient of the image.
    \item Non-maximum supression.
    \item Hysteresis thresholding.
\end{enumerate}
The noise reduction step was allready described in the \nameref{section:preprocessing} section of this paper.
\\\\
The intensity gradient of the image is calculated using the \textbf{Sobel Filter}:
\begin{equation}
    S(I(x,y)) :=\sqrt{(S_x*I(x,y))^2 + S_y(*I(x,y))^2}
\end{equation}
Generally we can define the gradient of the Image I as:
\begin{equation}
    G(I(x,y)) := \sqrt[]{I_x^2 +I_y^2}
\end{equation}
\noindent
In addition to the gradient we calculate the the orientation given by:
\begin{equation}
    \phi(x,y) = arctan(\frac{g_y}{g_x})
\end{equation}
 %todo
After getting the gradient magnitude and orientation of each pixel, the edges have
to be reduced to a thickness of one pixel which will be done with \textbf{non-maximum supression.}
%todo
\\\\
\noindent
To implement this methodologie in the document scanner, the function
which implements the full Canny edge detector\cite{opencv_canny} can be called.
\begin{center}
    \noindent
    $Canny(blurred\_image,75, 180,$\\
    $L2gradient = True, apertureSize = 3)$
\end{center}
\noindent
The functions input is the noise reduced image described in \nameref{section:preprocessing}.
For Hysteresis Threshold 75 for the lower-bound and 180 for the upper bound are chosen.
In several trials this settings could sufficiently provide the overall best results, 
when setting the aperture size to 3.
Instead of the predefined L1-Norm \cite{l1_norm}, the more precise L2-Norm\cite{l2_norm} is used.
\\\\
The canny edge detection filter returns an binary output image with edges
beeing set to 1 and other points being set to 0.
\\\\
This can be seen in Figure \ref{fig:canny}

\begin{figure}[H]
\centerline{\includegraphics[width=2.5in]{output/hoch_3_4_canny.jpg}}
\caption{The output image of the canny edge detection.}
\label{fig:canny}
\end{figure}

%%%%%%%%%%%%%%%%%%%%%%%%%%%%%%%%%%%%%%%%%%%%%%%%%%%%%%%%%%%%%%%%%%%%%%
\section{Corner Detection}
To find the corner of the document the previously generated edges 
in \nameref{section:edgedetection} have to be evaluated.
For this purpose, contours are formed which are represented by a curve that connects all 
continuous points with each other.\cite{SUZUKI198532}

The OpenCV function\cite{opencv_findcontours} is used as follows:
\begin{center}
    \noindent
    $contours, hierarchy = findContours(edges,$\\
    $RETR\_LIST, CHAIN\_APPROX\_SIMPLE)$
\end{center}

Where the input are the previously generated edges, the retrieval 
mode\cite{opencv_retrievalmode} which is 
set to retrieve the contours without establishing any hierarchical relationships 
and the contour approximation mode\cite{opencv_approxmode} which is set to ouput only their endopints.
(E.g: A rectengular contour is defined by the four corner points)
This can be seen in Figure \ref{fig:contours}

The assumption is made that the document is the largest rectangle in the image.
The contours are hierarchically sorted according to their area, with the largest polygon with four vertices being selected.

\begin{figure}[H]
\centerline{\includegraphics[width=2.5in]{output/hoch_3_5_contouredimage.jpg}}
\caption{The corners of the document.}
\label{fig:contours}
\end{figure}

% Appox Poly
\cite{doi:10.3138/FM57-6770-U75U-7727}
\cite{RAMER1972244}


% findContours
% Paper:
% https://reader.elsevier.com/reader/sd/pii/0734189X85900167?token=11C96FBB310332A0D41181ED5B29CA355D79630C93940FA6DCEABD7D12A580B1F46464823FDA2CB7D695CA063C30B005&originRegion=eu-west-1&originCreation=20220509094855
% contours = sorted(contours, key=contourArea, reverse=True)
%epsilon = 0.1*arcLength(contour,True)
%approx = approxPolyDP(contour,epsilon,True)


\section{Transformation}
\label{section:transformation}
% def order_points(pts):    
%     print(pts)
%     a = np.sum(np.square(pts[0,0]-pts[1,0]))
%     b = np.sum(np.square(pts[0,0]-pts[3,0]))

%     if a > b:
%         rect = pts.copy()
%         rect[0] = pts[0]
%         rect[1] = pts[3]
%         rect[2] = pts[2]
%         rect[3] = pts[1]
%         return rect
%     else:
%         rect = pts.copy()
%         rect[0] = pts[1]
%         rect[1] = pts[0]
%         rect[2] = pts[3]
%         rect[3] = pts[2]
%         return rect

% h = 3000
%     w = int(np.floor(h*(1/np.sqrt(2))))
%     dst_pts = np.array([[0, 0],   [w-1, 0],  [w-1, h-1], [0, h-1]], dtype=np.float32)
%     src_pts = np.array(order_points(doc_cnts), dtype=np.float32).reshape(4,2)
%     M = cv2.getPerspectiveTransform(src_pts, dst_pts)
% transformation = cv2.warpPerspective(grayscale_image, M, (w, h))

Before the transformation can be performed the orientation of the document in the image must be found out.
Without knowing the content of the document, it is not possible to know the real orientation. Therefore a few assumptions are made:
\begin{itemize}
    \item[1.] All documents are in DIN A4 format
    \item[2.] All provided documents are portrait format
    \item[3.] All provided documents are only rotated 90 degrees right or left, otherwise the document is scanned upside down.
\end{itemize}
These assumptions define the output of the scanner: A document in portrait format.
The first task is to find out if the document is landscape or portrait in the image.
Four examples of how the document could have been recorded can be seen in Figure \ref{fig:fourcorners}.
What is known from the list of the given four corners from the previous section is that the first element \bluecircled{-} is the most upper point in the image. 
Next, the distance of this point must be calculated with the second and fourth elements \redcircled{-} in the list and these two distances compared. For this purpose, the Euclidean distance calculation is used. If the distance to the second element is higher than to the fourth element, the document is recorded portrait. Else it was recorded landscape. 
Relative to the previous result, the list of found points must be put into this form:
\begin{center}
    \noindent
    $[\textrm{upper\_left}, \textrm{upper\_right}, \textrm{lower\_right}, \textrm{lower\_left}]$
\end{center}
This is used to calculate the perspective transformation from the given points to the destination points with the getPerspectiveTransform function from OpenCV\cite{opencv_getPerspectiveTransform}. The target points are the set points of the document in DIN A4 format.
\begin{center}
    \noindent
    $[[0,0], [\textrm{width},0], [\textrm{width},\textrm{height}], [0,\textrm{height}]]$\\
    $\textrm{height}=3000$\\
    $\textrm{width}=\frac{\textrm{height}}{\sqrt[2]{2}}$
\end{center}
\begin{figure}[H]
    \centerline{\includegraphics[width=2.5in]{output/a.png}}
    \caption{Four examples of the orientation of a scanned document}
    \label{fig:fourcorners}
\end{figure}
The last step of this section is to apply the perspective transformation to the grayscale image. 
For this purpose the warpPerspective function from OpenCV\cite{opencv_warpPerspective} is used:
\begin{center}
    \noindent
    $transformation = cv2.warpPerspective(grayscale\_image, M, (width, height))$
\end{center}
The function uses the grayscale image, the calculated perspective Transformation(M), and the size of the 
output image to perform the transformation. The optional parameters are set to default.
The output of the transformation can be seen in Figure \ref{fig:transformation}.

\begin{figure}[H]
    \centerline{\includegraphics[width=2.5in]{output/hoch_3_6_transformation.jpg}}
    \caption{Document after the transformation.}
    \label{fig:transformation}
\end{figure}




\section{Binary Image}
% adaptive_binary_image = adaptiveThreshold(transformation,255,ADAPTIVE_THRESH_GAUSSIAN_C, THRESH_BINARY,11,6)
%tresh, binary_image = threshold(transformation, 100,255,THRESH_BINARY)


\section{Summary}
% Ergebnisse mit beispielen
The example within the report and the two other examples attached below (Figures \ref{fig:schraeg2_original},\ref{fig:schraeg2_binary},\ref{fig:hoch1_original},\ref{fig:hoch1_binary}) show that the scanner does work even if the document is recorded in portrait or landscape format. The pictures of the documents show a small problem, that the recorded documents were wavy. Therefore the output documents show a wavy edge, but the content is still perfectly readable. The image of document of Figure \ref{fig:schraeg2_original} is even worse readable than the transformed document seen in Figure \ref{fig:schraeg2_binary}.\\\\
For further optimization it would be possible to add a character recognition to find out the real orientation of the document. This would fix the cases when the document has landscape format or is upside down and the assumptions 2 and 3 in the \nameref{section:transformation} section didn't have to be made.


\begin{figure}[H]
    \centerline{\includegraphics[width=2.5in]{output/schraeg_2_1_original.jpg}}
    \caption{Document recorded in landscape.}
    \label{fig:schraeg2_original}
\end{figure}

\begin{figure}[H]
    \centerline{\includegraphics[width=2.5in]{output/schraeg_1_8_adaptive_binary_image_mean.jpg}}
    \caption{Document recorded in landscape transformed into binary document}
    \label{fig:schraeg2_binary}
\end{figure}

\begin{figure}[H]
    \centerline{\includegraphics[width=2.5in]{output/hoch_1_1_original.jpg}}
    \caption{Document recorded in landscape.}
    \label{fig:hoch1_original}
\end{figure}

\begin{figure}[H]
    \centerline{\includegraphics[width=2.5in]{output/hoch_1_8_adaptive_binary_image_mean.jpg}}
    \caption{Another document example transformed into binary document}
    \label{fig:hoch1_binary}
\end{figure}
















% %%%%%%%%%%%%%%%%%%%%%%%%%%%%%%%%%%%%%%%%%%%%%%%%%%%%%%%%%%%%%%%%%%%%%%
% %%%%%%%%%%%%%%%%%%%%%%%%%%%%%%%%%%%%%%%%%%%%%%%%%%%%%%%%%%%%%%%%%%%%%%
% %%%%%%%%%%%%%%%%%%%%%%%%%%%%%%%%%%%%%%%%%%%%%%%%%%%%%%%%%%%%%%%%%%%%%%
% %%%%%%%%%%%%%%%%%%%%%%%%%%%%%%%%%%%%%%%%%%%%%%%%%%%%%%%%%%%%%%%%%%%%%%
% \section{Transformation}
% \footnotetext{Examine the input file, asme2ej.tex, to see how a footnote is given in a head.}

% Footnotes are referenced with superscript numerals and are numbered consecutively from 1 to the end of the paper\footnote{Avoid footnotes if at all possible.}. Footnotes should appear at the bottom of the column in which they are referenced.


% %%%%%%%%%%%%%%%%%%%%%%%%%%%%%%%%%%%%%%%%%%%%%%%%%%%%%%%%%%%%%%%%%%%%%%
% \section{Mathematics}

% Equations should be numbered consecutively beginning with (1) to the end of the paper, including any appendices.  The number should be enclosed in parentheses and set flush right in the column on the same line as the equation.  An extra line of space should be left above and below a displayed equation or formula. \LaTeX\ can automatically keep track of equation numbers in the paper and format almost any equation imaginable. An example is shown in Eqn.~(\ref{eq_ASME}). The number of a referenced equation in the text should be preceded by Eqn.\ unless the reference starts a sentence in which case Eqn.\ should be expanded to Equation.

% \begin{equation}
% f(t) = \int_{0_+}^t F(t) dt + \frac{d g(t)}{d t}
% \label{eq_ASME}
% \end{equation}

% %%%%%%%%%%%%%%%%%%%%%%%%%%%%%%%%%%%%%%%%%%%%%%%%%%%%%%%%%%%%%%%%%%%%%%
% \section{Figures}
% \label{sect_figure}

% All figures should be positioned at the top of the page where possible.  All figures should be numbered consecutively and centered under the figure as shown in Fig.~\ref{figure_ASME}. All text within the figure should be no smaller than 7~pt. There should be a minimum two line spaces between figures and text. The number of a referenced figure or table in the text should be preceded by Fig.\ or Tab.\ respectively unless the reference starts a sentence in which case Fig.\ or Tab.\ should be expanded to Figure or Table.


% %%%%%%%%%%%%%%%%%%%%%%%%%%%%%%%%%%%%%%%%%%%%%%%%%%%%%%%%%%%%%%%%%%%%%%
% %%%%%%%%%%%%%%%% begin figure %%%%%%%%%%%%%%%%%%%
% \begin{figure}[t]
% \begin{center}
% \setlength{\unitlength}{0.012500in}%
% \begin{picture}(115,35)(255,545)
% \thicklines
% \put(255,545){\framebox(115,35){}}
% \put(275,560){Beautiful Figure}
% \end{picture}
% \end{center}
% \caption{The caption of a single sentence does not have period at the end}
% \label{figure_ASME} 
% \end{figure}
% %%%%%%%%%%%%%%%% end figure %%%%%%%%%%%%%%%%%%% 
% %%%%%%%%%%%%%%%%%%%%%%%%%%%%%%%%%%%%%%%%%%%%%%%%%%%%%%%%%%%%%%%%%%%%%%

% In the following subsections, I have inserted figures that have been provided by authors in order to demonstrate what to avoid.  In each case the authors provided figures that are 3.25in wide and 600dpi in the .tif graphics format.  The papers containing these figures have been held from production due to their poor quality. 

% %%%%%%%%%%%%%%%%%%%%%%%%%%%%%%%%%%%%%%%%%%%%%%%%%%%%%%%%%%%%%%%%%%%%%%
% \subsection{The 1st Example of Bad Figure}

% %%%%%%%%%%%%%%%% begin figure %%%%%%%%%%%%%%%%%%%
% %%% 3.34in is the maximum width you can have for a figure
% \begin{figure} 
% \centerline{\includegraphics[width=3.34in]{figure/FMANU_MD_05_1107_11.jpg}}
% \caption{Example taken from a paper that was held from production because the image quality is poor.  ASME sets figures captions in 8pt, Helvetica Bold.}
% \label{fig_example1.jpg}
% \end{figure}
% %%%%%%%%%%%%%%%% end figure %%%%%%%%%%%%%%%%%%%

% In order to place the figure in this template using MSWord, select Insert Picture from File, and use wrapping that is top and bottom. Make sure the figure is 3.25in wide.
 
% Figure~`\ref{fig_example1.jpg}
% was taken from a recent paper that was held from publication, because the text is fuzzy and unreadable. It was probably obtained by taking a screen shot of the computer output of the authors software. This means the original figure was 72dpi (dots per inch) on a computer screen. There is no way to improve the quality such a low resolution figure.
 
% In order to understand how poor the quality of this figure is, please zoom in slightly, say to 200\%.  Notice that while the font of the paper is clear at this size, the font in the figures is fuzzy and blurred.  It is impossible to make out the small symbol beside the numbers along the abscissa of the graph.  Now consider the labels Time and Cost. They are clearly in fonts larger that the text of the article, yet the pixilation or rasterization, associated with low resolution is obvious. This figure must be regenerated at higher resolution to ensure quality presentation.

% The poor quality of this figure is immediately obvious on the printed page, and reduces the impact of the research contribution of the paper, and in fact detracts from the perceived quality of the journal itself.



% %%%%%%%%%%%%%%%%%%%%%%%%%%%%%%%%%%%%%%%%%%%%%%%%%%%%%%%%%%%%%%%%%%%%%%
% \subsection{The 2nd Example of Bad Figure}

% %%%%%%%%%%%%%%%% begin figure %%%%%%%%%%%%%%%%%%%
% \begin{figure} 
% \centerline{\includegraphics[width=3.34in]{figure/FMANU_MD_05_1272_5.jpg}}
% \caption{While this figures is easily readable at a double column width of 6.5in, when it is shrunk to 3.25in column width the text is unreadable. This paper was held from production.}
% \label{fig_example2.jpg}
% \end{figure}
% %%%%%%%%%%%%%%%% end figure %%%%%%%%%%%%%%%%%%%

% Figure~\ref{fig_example2.jpg}
% demonstrates a common problem that arises when a figure is scaled down fit a single column width of 3.25in.  The original figure had labels that were readable at full size, but become unreadable when scaled to half size.  This figure also suffers from poor resolution as is seen in the jagged lines the ovals that form the chain.

% This problem can be addressed by increasing the size of the figure to a double column width of 6.5in, so the text is readable.  But this will not improve the line pixilation, and a large low resolution figure is less desirable than a small one.  This also significantly expands the length of the paper, and may cause it to exceed the JMD nine page limit.  Additional pages require page charges of \$200 per page.  It is best to regenerate the figure at the resolution that ensures a quality presentation.


% %%%%%%%%%%%%%%%%%%%%%%%%%%%%%%%%%%%%%%%%%%%%%%%%%%%%%%%%%%%%%%%%%%%%%%
% \subsection{The 3rd Example of Bad Figure}
% %%%%%%%%%%%%%%%% begin figure %%%%%%%%%%%%%%%%%%%
% \begin{figure} 
% \centerline{\includegraphics[width=3.25in]{figure/FMANU_MD_04_1274_13.jpg}}
% \caption{Another example of a figure with unreadable text.  Even when the paper was expanded to double column width the text as shown in Fig.~\ref{fig_example4.jpg} was of such low quality that the paper was held from production.}
% \label{fig_example3.jpg}
% \end{figure}
% %%%%%%%%%%%%%%%% end figure %%%%%%%%%%%%%%%%%%%

% %%%%%%%%%%%%%%%% begin figure %%%%%%%%%%%%%%%%%%%
% %%% the maximum width in double column is 6.85in
% \begin{figure*} 
% \centerline{\includegraphics[width=6.85in]{figure/FMANU_MD_04_1274_13.jpg}}
% \caption{A figure expanded to double column width the text from Figure~\ref{fig_example3.jpg}}
% \label{fig_example4.jpg}
% \end{figure*}
% %%%%%%%%%%%%%%%% end figure %%%%%%%%%%%%%%%%%%%
% An author provided the high resolution image 
% in Fig.~\ref{fig_example3.jpg}
% that was sized to a single column width of 3.25in.  Upon seeing the poor quality of the text, the publisher scaled the image to double column width as shown in Fig.~\ref{fig_example4.jpg} 
% at which point it took half of a page.  The publisher went on to do this for all eight figures generating four pages of figures that the author did not expect. ASME stopped production of the paper even with the larger figures due to the pixilation of the font.

% Clearly the text in this figure is unreadable, and it is doubtful that the author can print the output in a way that it is readable.  This is a problem that the author must solve, not the publisher. 

% As you might expect, I have many more examples, but in the end the author is the best judge of what is needed in each figure.  ASME simply requires that the image meet a minimum standard for font and line quality, specifically the font should be the appropriate size and not be blurred or pixilated, and that lines should be the appropriate weight and have minimal, preferably no, pixilation or rasterization.


% %%%%%%%%%%%%%%%%%%%%%%%%%%%%%%%%%%%%%%%%%%%%%%%%%%%%%%%%%%%%%%%%%%%%%%
% \section{Tables}

% %%%%%%%%%%%%%%%%%%%%%%%%%%%%%%%%%%%%%%%%%%%%%%%%%%%%%%%%%%%%%%%%%%%%%%
% %%%%%%%%%%%%%%% begin table   %%%%%%%%%%%%%%%%%%%%%%%%%%
% \begin{table}[t]
% \caption{Figure and table captions do not end with a period}
% \begin{center}
% \label{table_ASME}
% \begin{tabular}{c l l}
% & & \\ % put some space after the caption
% \hline
% Example & Time & Cost \\
% \hline
% 1 & 12.5 & \$1,000 \\
% 2 & 24 & \$2,000 \\
% \hline
% \end{tabular}
% \end{center}
% \end{table}
% %%%%%%%%%%%%%%%% end table %%%%%%%%%%%%%%%%%%% 
% %%%%%%%%%%%%%%%%%%%%%%%%%%%%%%%%%%%%%%%%%%%%%%%%%%%%%%%%%%%%%%%%%%%%%%

% All tables should be numbered consecutively  and centered above the table as shown in Table~\ref{table_ASME}. The body of the table should be no smaller than 7 pt.  There should be a minimum two line spaces between tables and text.


% %%%%%%%%%%%%%%%%%%%%%%%%%%%%%%%%%%%%%%%%%%%%%%%%%%%%%%%%%%%%%%%%%%%%%%
% \section{Citing References}

% %%%%%%%%%%%%%%%%%%%%%%%%%%%%%%%%%%%%%%%%%%%%%%%%%%%%%%%%%%%%%%%%%%%%%%
% The ASME reference format is defined in the authors kit provided by the ASME.  The format is:

% \begin{quotation}
% {\em Text Citation}. Within the text, references should be cited in  numerical order according to their order of appearance.  The numbered reference citation should be enclosed in brackets.
% \end{quotation}

% The references must appear in the paper in the order that they were cited.  In addition, multiple citations (3 or more in the same brackets) must appear as a `` [1-3]''.  A complete definition of the ASME reference format can be found in the  ASME manual \cite{asmemanual}.

% The bibliography style required by the ASME is unsorted with entries appearing in the order in which the citations appear. If that were the only specification, the standard {\sc Bib}\TeX\ unsrt bibliography style could be used. Unfortunately, the bibliography style required by the ASME has additional requirements (last name followed by first name, periodical volume in boldface, periodical number inside parentheses, etc.) that are not part of the unsrt style. Therefore, to get ASME bibliography formatting, you must use the \verb+asmems4.bst+ bibliography style file with {\sc Bib}\TeX. This file is not part of the standard BibTeX distribution so you'll need to place the file someplace where LaTeX can find it (one possibility is in the same location as the file being typeset).

% With \LaTeX/{\sc Bib}\TeX, \LaTeX\ uses the citation format set by the class file and writes the citation information into the .aux file associated with the \LaTeX\ source. {\sc Bib}\TeX\ reads the .aux file and matches the citations to the entries in the bibliographic data base file specified in the \LaTeX\ source file by the \verb+\bibliography+ command. {\sc Bib}\TeX\ then writes the bibliography in accordance with the rules in the bibliography .bst style file to a .bbl file which \LaTeX\ merges with the source text.  A good description of the use of {\sc Bib}\TeX\ can be found in \cite{latex, goosens} (see how two references are handled?).  The following is an example of how three or more references \cite{latex, asmemanual,  goosens} show up using the \verb+asmems4.bst+ bibliography style file in conjunction with the \verb+asme2ej.cls+ class file. Here are some more \cite{art, blt, ibk, icn, ips, mts, mis, pro, pts, trt, upd} which can be used to describe almost any sort of reference.

% %%%%%%%%%%%%%%%%%%%%%%%%%%%%%%%%%%%%%%%%%%%%%%%%%%%%%%%%%%%%%%%%%%%%%%
% \section{Conclusions}
% The only way to ensure that your figures are presented in the ASME Journal of Mechanical Design in the way you feel is appropriate and meets the requirement for quality presentation is for you to prepare a double column version of the paper in a form similar to that used by the Journal.

% This gives you the opportunity to ensure that the figures are sized appropriately, in particular that the labels are readable and match the size of the text in the journal, and that the line weights and resolutions have no pixilation or rasterization.  Poor quality figures are immediately obvious on the printed page, and this detracts from the perceived quality of the journal.

% I am pleased to provide advice on how to improve any figure, but this effort must start with a two-column version of the manuscript. Thank you in advance for your patience with this effort, it will ensure quality presentation of your research contributions.



% %%%%%%%%%%%%%%%%%%%%%%%%%%%%%%%%%%%%%%%%%%%%%%%%%%%%%%%%%%%%%%%%%%%%%%
% \section{Discussions}
% This template is not yet ASME journal paper format compliant at this point.
% More specifically, the following features are not ASME format compliant.
% \begin{enumerate}
% \item
% The format for the title, author, and abstract in the cover page.
% \item
% The font for title should be 24 pt Helvetica bold.
% \end{enumerate}

% \noindent
% If you can help to fix these problems, please send us an updated template.
% If you know there is any other non-compliant item, please let us know.
% We will add it to the above list.
% With your help, we shall make this template 
% compliant to the ASME journal paper format.


% %%%%%%%%%%%%%%%%%%%%%%%%%%%%%%%%%%%%%%%%%%%%%%%%%%%%%%%%%%%%%%%%%%%%%%
% \begin{acknowledgment}
% ASME Technical Publications provided the format specifications for the Journal of Mechanical Design, though they are not easy to reproduce.  It is their commitment to ensuring quality figures in every issue of JMD that motivates this effort to have authors review the presentation of their figures.  

% Thanks go to D. E. Knuth and L. Lamport for developing the wonderful word processing software packages \TeX\ and \LaTeX. We would like to thank Ken Sprott, Kirk van Katwyk, and Matt Campbell for fixing bugs in the ASME style file \verb+asme2ej.cls+, and Geoff Shiflett for creating 
% ASME bibliography stype file \verb+asmems4.bst+.
% \end{acknowledgment}

% %%%%%%%%%%%%%%%%%%%%%%%%%%%%%%%%%%%%%%%%%%%%%%%%%%%%%%%%%%%%%%%%%%%%%%
% % The bibliography is stored in an external database file
% % in the BibTeX format (file_name.bib).  The bibliography is
% % created by the following command and it will appear in this
% % position in the document. You may, of course, create your
% % own bibliography by using thebibliography environment as in
% %
% % \begin{thebibliography}{12}
% % ...
% % \bibitem{itemreference} D. E. Knudsen.
% % {\em 1966 World Bnus Almanac.}
% % {Permafrost Press, Novosibirsk.}
% % ...
% % \end{thebibliography}

% % Here's where you specify the bibliography style file.
% % The full file name for the bibliography style file 
% % used for an ASME paper is asmems4.bst.
% \bibliographystyle{asmems4}
\bibliographystyle{alpha}

% Here's where you specify the bibliography database file.
% The full file name of the bibliography database for this
% article is asme2e.bib. The name for your database is up
% to you.
\bibliography{asme2e}

%%%%%%%%%%%%%%%%%%%%%%%%%%%%%%%%%%%%%%%%%%%%%%%%%%%%%%%%%%%%%%%%%%%%%%
% \appendix       %%% starting appendix
% \section*{Appendix A: Head of First Appendix}
% Avoid Appendices if possible.

% %%%%%%%%%%%%%%%%%%%%%%%%%%%%%%%%%%%%%%%%%%%%%%%%%%%%%%%%%%%%%%%%%%%%%%
% \section*{Appendix B: Head of Second Appendix}
% \subsection*{Subsection head in appendix}
% The equation counter is not reset in an appendix and the numbers will
% follow one continual sequence from the beginning of the article to the very end as shown in the following example.
% \begin{equation}
% a = b + c.
% \end{equation}

\end{document}
